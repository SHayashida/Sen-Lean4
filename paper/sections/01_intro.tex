\section{Introduction}

Classical impossibility theorems explain why certain axiom sets cannot hold simultaneously, but engineering workflows also need the complementary question: \emph{which minimal lever changes restore feasibility}.
This paper frames that question as an auditable systems problem.

Our central thesis is practical: a small, proof-carrying atlas can already provide publishable insight if it couples (i) strict reproducibility, (ii) explicit safety assumptions, and (iii) boundary explanation artifacts.
We operationalize this thesis in sen24 with five claim blocks (C1--C5) mapped to concrete commands and artifacts in \texttt{docs/paper\_claims\_map.md}.

Contributions are systems-oriented:
\begin{itemize}[leftmargin=*]
  \item A Spec/Encode/Solve/Check architecture that keeps solver heuristics separate from trusted checking.
  \item Frontier identification with MUS/MCS for UNSAT explanations.
  \item Proof-carrying UNSAT workflow (CNF+LRAT+Lean).
  \item SAT Gallery extraction with solver-independent witness validation.
  \item Lightweight but strict reproducibility gates in CI.
\end{itemize}
